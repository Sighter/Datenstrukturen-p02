

\documentclass{article}
\usepackage{times}
\usepackage{german}
\usepackage{fancyhdr}
\usepackage[utf8]{inputenc}

\title{Datenstrukturen\\Praxiseinheit 2}
\author
{
	Sascha Ebert\\
	MatNr: 177182\\
	\texttt{sascha.ebert@s2006.tu-chemnitz.de}\\
	\texttt{https://github.com/Sighter/Datenstrukturen-p02}\\
}

\date{\today}

\begin{document}
\maketitle

\begin{abstract}
Anbei finden sie Erläuterungen zu den Aufgaben der 2. Praxiseinheit.
Alle anderen Programmcode-bezogenen Aufgaben finden sie im Quelltext an den jeweiligen Positionen
erläutert. Es somit auch möglich das oben vermerkte Git-Repository zu nutzen um den Quelltext
zu betrachten. Die aufgabenbezogenen Kommentare sind mit `Exercise' oder `Hint' gekennzeichnet.
\end{abstract}

\section*{Aufgabenzuordnung}
\begin{tabular}{ c l l }
  Aufgabe & Datei & Funktion/Methode\\
  \hline
  1  & network.h & Komplett\\
  2  & network.cpp & network::find\_node\\
  3  & network.cpp & network::pac\_hunter\\
\end{tabular}

\section*{Anmerkungen}
Alle printf, random Functionsaufrufe und die damit verbundenen Include-Anweisungen wurden
auskommentiert.

\end{document}


